%% LyX 2.3.4.2 created this file.  For more info, see http://www.lyx.org/.
%% Do not edit unless you really know what you are doing.
\documentclass[english,dvipsnames,aspectratio=169]{beamer}
\usepackage{mathptmx}
\usepackage{eulervm}
\usepackage[T1]{fontenc}
\usepackage[latin9]{inputenc}
\usepackage{babel}
\usepackage{amstext}
\usepackage{amssymb}
\usepackage{graphicx}
\usepackage{ifthen}
\usepackage{xcolor}
\usepackage{xspace}
\usepackage{tikz}
\usetikzlibrary{tikzmark}
\usetikzlibrary{calc}
\usepackage{pgfplots}
%\pgfplotsset{compat=1.17}
\usepackage{booktabs}
\usepackage{xpatch}

\xpatchcmd{\itemize}
  {\def\makelabel}
  {\ifnum\@itemdepth=1\relax
     \setlength\itemsep{2ex}% separation for first level
   \else
     \ifnum\@itemdepth=2\relax
       \setlength\itemsep{1ex}% separation for second level
     \else
       \ifnum\@itemdepth=3\relax
         \setlength\itemsep{0.5ex}% separation for third level
   \fi\fi\fi\def\makelabel
  }
 {}
 {}

\ifx\hypersetup\undefined
  \AtBeginDocument{%
    \hypersetup{unicode=true,pdfusetitle,
 bookmarks=true,bookmarksnumbered=false,bookmarksopen=false,
 breaklinks=false,pdfborder={0 0 0},pdfborderstyle={},backref=false,colorlinks=true,
 allcolors=NYUPurple,urlcolor=LightPurple}
  }
\else
  \hypersetup{unicode=true,pdfusetitle,
 bookmarks=true,bookmarksnumbered=false,bookmarksopen=false,
 breaklinks=false,pdfborder={0 0 0},pdfborderstyle={},backref=false,colorlinks=true,
 allcolors=NYUPurple,urlcolor=LightPurple}
\fi

\makeatletter

%%%%%%%%%%%%%%%%%%%%%%%%%%%%%% LyX specific LaTeX commands.
%% Because html converters don't know tabularnewline
\providecommand{\tabularnewline}{\\}

%%%%%%%%%%%%%%%%%%%%%%%%%%%%%% Textclass specific LaTeX commands.
% this default might be overridden by plain title style
\newcommand\makebeamertitle{\frame{\maketitle}}%
% (ERT) argument for the TOC
\AtBeginDocument{%
  \let\origtableofcontents=\tableofcontents
  \def\tableofcontents{\@ifnextchar[{\origtableofcontents}{\gobbletableofcontents}}
  \def\gobbletableofcontents#1{\origtableofcontents}
}

%%%%%%%%%%%%%%%%%%%%%%%%%%%%%% User specified LaTeX commands.
\usetheme{CambridgeUS} 
\beamertemplatenavigationsymbolsempty


% Set Color ==============================
\definecolor{NYUPurple}{RGB}{87,6,140}
\definecolor{LightPurple}{RGB}{165,11,255}


\setbeamercolor{title}{fg=NYUPurple}
\setbeamercolor{frametitle}{fg=NYUPurple}

\setbeamercolor{background canvas}{fg=NYUPurple, bg=white}
\setbeamercolor{background}{fg=black, bg=NYUPurple}

\setbeamercolor{palette primary}{fg=black, bg=gray!30!white}
\setbeamercolor{palette secondary}{fg=black, bg=gray!20!white}
\setbeamercolor{palette tertiary}{fg=gray!20!white, bg=NYUPurple}

\setbeamertemplate{headline}{}
\setbeamerfont{itemize/enumerate body}{}
\setbeamerfont{itemize/enumerate subbody}{size=\normalsize}

\setbeamercolor{parttitle}{fg=NYUPurple}
\setbeamercolor{sectiontitle}{fg=NYUPurple}
\setbeamercolor{sectionname}{fg=NYUPurple}
\setbeamercolor{section page}{fg=NYUPurple}
%\setbeamercolor{description item}{fg=NYUPurple}
%\setbeamercolor{block title}{fg=NYUPurple}

\setbeamertemplate{blocks}[rounded][shadow=false]
\setbeamercolor{block body}{bg=normal text.bg!90!NYUPurple}
\setbeamercolor{block title}{bg=NYUPurple!30, fg=NYUPurple}



\AtBeginSection[]{
  \begin{frame}
  \vfill
  \centering
\setbeamercolor{section title}{fg=NYUPurple}
 \begin{beamercolorbox}[sep=8pt,center,shadow=true,rounded=true]{title}
    \usebeamerfont{title}\usebeamercolor[fg]{title}\insertsectionhead\par%
  \end{beamercolorbox}
  \vfill
  \end{frame}
}

\makeatother

\setlength{\parskip}{\medskipamount} 

\input ../macros

\begin{document}
\input ../rosenberg-macros

\title[DS-GA 1003]{Course Overview}
\author{He He}
\date{\today}
\institute{CDS, NYU}

\makebeamertitle
\mode<article>{Just in article version}

\begin{frame}{Contents}
\tableofcontents{}
\end{frame}

\section{Logistics}
\begin{frame}{Course Staff}
\begin{itemize}
\item Instructors:
    \begin{itemize}
        \item He He (CS/CDS): most lectures
        \item Marylou Gabrie (CDS): 2 lectures and assignments
        \item Ming Liao (NYUSH): in-person sections in Shanghai
    \end{itemize}

\item Section leaders:
    \begin{itemize}
        \item \underline{Xiangyun Chu}, Jatin Khilnani, \underline{Xiaocheng Li}, Haresh Rengaraj Rajamohan, \underline{Daeyoung Kim}
    \end{itemize}

\item Graders:
    \begin{itemize}
        \item Artie Shen, \underline{Daeyoung Kim}, \underline{Xiangyun Chu}, \underline{Xiaocheng Li}, Aniket Bhatnagar, Udit Arora
    \end{itemize}
\end{itemize}
\end{frame}

\begin{frame}{Logistics}
\begin{itemize}
\item Class webpage: \url{https://nyu-ds1003.github.io/spring2021} 
\begin{itemize}
\item Course material will be distributed on the website
\end{itemize}

\item Piazza: \url{piazza.com/nyu/spring2021/dsga1003}
\begin{itemize}
\item \textbf{All class announcements via Piazza}
\item Ask all questions on Piazza
\end{itemize}

\item Gradescope: entry code \textbf{ZRWBGD}
\begin{itemize}
\item Sign up yourself for assignment submission 
\end{itemize}

\item Office Hours:
\begin{itemize}
\item All office hours will be on Zoom
\item See course calender for details: \url{https://tinyurl.com/y23zkucf} 
\end{itemize}
\end{itemize}
\end{frame}

\begin{frame}{Evaluation}
\begin{itemize}
\item 7 assignments ($1\times 4\% + 6 \times 6\% = 40\%$)
\item Two tests ($60\%$)
\begin{itemize}
\item Midterm Exam ($30\%$) in Week 8 (March 23rd), covering material up to Week 7
\item Final Exam ($30\%$), time TBD, covering all material
\end{itemize}

\item These scores determine ``class rank''.

\item Typical grade distribution: A (40\%), A- (20\%), B+ (20\%), B (10\%),
B- (5\%), <B- (5\%)
\end{itemize}
\end{frame}
%
\begin{frame}{Homework}
\begin{itemize}
    \item Assignment 0: Help you get farmiliar with the format (not submitted or graded)
    \item First assignment out now -- due on \textbf{Feb 10}
    \item Submit through Gradescope as a \textbf{PDF document}
    \item Late policy: Assignments are accepted up to \textbf{48 hours} late (see more details on website)

\item Collaboration is fine, but
\begin{itemize}
\item Write up solutions and code on your own
\item List names of who you talked to about each problem
\end{itemize}

\end{itemize}
\end{frame}

\begin{frame}
    {Exams (60\%)}
    \begin{itemize}
        \item Exams will be submitted through Gradescope (similar to assignments)
        \item Start within \textbf{24 hours} once it's released
        \item Submit in \textbf{2 hours} once started
        \item Typesetting or handwritten, but must be submitted in \textbf{PDF}
        \item No collaboration is allowed
        \item Exams from previous years will be posted
    \end{itemize}
\end{frame}

\begin{frame}{Prerequisites}
\begin{itemize}
\item DS-GA 1001: Introduction to Data Science 
\item DS-GA 1002: Statistical and Mathematical Methods
\item Math
\begin{itemize}
\item Multivariate Calculus
\item Linear Algebra
\item Probability Theory
\item Statistics
\item {[}Preferred{]} Proof-based linear algebra or real analysis
\end{itemize}
\item Python programming (numpy)
\end{itemize}
\end{frame}


\section{Course Overview and Goals}
\begin{frame}{Syllabus (Tentative)}

13 weeks of instruction + 1 week midterm exam
\begin{itemize}
    \item 2 weeks: introduction to \textbf{statistical learning theory}, \textbf{optimization}

\item 2-3 weeks: \textbf{Linear }methods for binary classification
and regression (also\textbf{ kernel methods)}

\item 2 weeks: \textbf{Probabilistic models}, \textbf{Bayesian}
methods

\item 1 week: \textbf{Multiclass} classification and introduction to \textbf{structured
prediction}

\item 3-4 weeks: \textbf{Nonlinear} methods (\textbf{trees}, \textbf{ensemble}
methods, and \textbf{neural networks})

\item 2 weeks: \textbf{Unsupervised} learning: \textbf{clustering} and \textbf{latent variable} models

\item See tentative schedule on the webpage

\item Applications and practical algorithms may be covered in labs
\end{itemize}
\end{frame}
%
\begin{frame}{High Level Goals of the Class}
\begin{itemize}
\item Learn fundamental building blocks of machine learning
\item Goal is to start seeing\\
\begin{itemize}
\item[] \textbf{fancy new method A ``is just'' familiar thing B + familiar
thing C + tweak D}
\end{itemize}
        \begin{itemize}
\item SVM ``\textbf{is just}'' ERM with hinge loss with $\ell_{2}$ regularization
\item Pegasos ``\textbf{is just}'' SVM with SGD with a particular step
size rule
\item Random forest ``\textbf{is just}'' bagging with trees, with an interesting
tweak on choosing splitting variables
        \end{itemize}
\end{itemize}
\end{frame}
%
\begin{frame}{Level of the Class}
\begin{itemize}
\item We will learn how to build all ML algorithms \textbf{from scratch}
-- no ML libraries, just numpy.
\item Once we have built it from scratch once, we can use the sklearn version.
\end{itemize}
\end{frame}
\end{document}
